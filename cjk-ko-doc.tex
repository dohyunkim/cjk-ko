%% $Id: cjk-ko-doc.tex,v 1.10 2013/02/26 01:40:15 nomos Exp $
%% public domain

\documentclass[dvipdfmx,b5paper]{article}
\usepackage{geometry}
\usepackage[cjk,hangul,usedotemph]{kotex}
\usepackage{xcolor}
\usepackage{url}

\def\cs#1{\texttt{\color{teal}\char92 #1}}
\def\koTeX{k$o$.\kern-.1667em\TeX}
\def\cjkko{\mbox{CJK-k\kern-.1em\textit{o}}}

\linespread{1.3}

\title{\cjkko\ 간단 매뉴얼}
\author{김도현\quad \texttt{nomos at ktug org}}
\date{Version 1.1\quad\today}
\begin{document}
\maketitle

\begin{abstract}
  For introduction in English, please see \verb|README| file in this package.

  본래 \TeX\ Writer라는 iOS 앱을 위해 만들었던 한글 패키지를 \TeX\ Live용으로
  수정했다. 특히 \TeX\ Live 버전에서는 나눔글꼴을 트루타입 대신 Type1
  글꼴---물론 subfont들이다---로 변환해 넣음으로써 다양한 DVI 툴을 이용할 수
  있게 했다.
\end{abstract}

\section{Introduction}
  \begin{itemize}
    \item CJK 패키지의 \texttt{UTF8} 환경 이용
    \item 복잡한 환경 지시 없이 \texttt{kotex}만 부르면 바로 한글 가능
      \begin{itemize}
	\item[] \hskip2cm \fbox{\vtop{\hsize=.5\textwidth\baselineskip=1.1em
	    \cs{begin\{document\}}\par
	    \cs{begin\{CJK\}\{UTF8\}\{mj\}}\par
	    \leavevmode\llap{No More $\rightarrow$\quad}\quad $\cdots$\par
	    \cs{end\{CJK\}}\par
	    \cs{end\{document\}}}}
      \end{itemize}
    \item 한글 문서에 최적화된 줄바꿈 기능~--- 예: 괄호 앞뒤, 수식 뒤
    \item 영문자와 한글의 조화 추구~--- 예: 한글 글자 크기 조정 허용
    \item 기초적인 자동조사 기능
    \item 오로지 \dotemph{현대 한국어} 문서를 위한 패키지~---
      중세한글, 일본어, 중국어는 지원하지 않는다.
  \end{itemize}

\section{Package options}
  \begin{description}
    \item[불러오기]: \quad\cs{usepackage[cjk]\{kotex\}}
      \medskip
    \item[패키지 옵션]:
      \begin{itemize}
	\item[\texttt{cjk}] \TeX\ Live에선 이 옵션이 없으면 kotexutf\,가
	  로드된다. 단, \verb|kotexutf.sty| 파일을 찾을 수 없다면
	  이 옵션이 없더라도 \cjkko\ 패키지를 부른다.
	\item[\texttt{hangul}] 한글 캡션, 줄 간격, 단어 간격, frenchspacing
	  등의 조정이 이루어진다. 문서의 주된 언어가 한글이라고 선언하는 옵션
	\item[\texttt{hanja}] \verb|[hangul]| $+$ 한자 캡션
	\item[\texttt{nojosa}] 자동조사 기능 끄기. 이 옵션을 주더라도
	  자동조사 명령이 에러를 내는 건 아니다.
	\item[\texttt{usedotemph}] \cs{dotemph} 명령 허용\\
	  --- 이 옵션은 ulem 패키지도 부르므로 \uline{밑줄 긋기} 가능
	\item[\texttt{usecjkt1font}] 영문자도 한글 글꼴---나눔폰트---로 식자.
	   라틴 알파벳이 거의 없는 소설책 따위에 유용할 수 있다.
      \end{itemize}
  \end{description}

\section{User commands}
  \begin{description}
    \item[\cs{CJKscale}] 한글만 글자크기 조정
      \begin{itemize}\leftskip-1cm
	\item 예: \cs{CJKscale\{0.95\}}
	\item \verb|[usecjkt1font]| 옵션과는 같이 쓸 수 없다.
	\item \verb|[hangul]| 옵션 아래서는 단어 간격, 줄 간격,
	  들여쓰기 크기도 자동 조정
	\item 전처리부에서만 쓸 수 있다.
      \end{itemize}
    \item[\cs{lowerCJKchar}] 한글만 아래로 끌어내려 식자
      \begin{itemize}\leftskip-1cm
	\item 예: \cs{lowerCJKchar\{-0.07em\} \% 끌어올려 식자}
	\item \verb|[usecjkt1font]| 옵션과는 같이 쓸 수 없다.
	\item 전처리부에서만 쓸 수 있다.
      \end{itemize}
    \item[\cs{dotemph}] \dotemph{드러냄표}
      \begin{itemize}\leftskip-1cm
	\item 예: \cs{dotemph\{드러냄표\}}
	\item \verb|[usedotemph]| 옵션 아래에서만 쓸 수 있다.
	\item \koTeX 과 마찬가지로 \cs{dotemphraise} \cs{dotemphchar} 명령
	  재정의 가능
      \end{itemize}
    \item[기타] 사용자 명령은 CJK 패키지 문서를 참조
  \end{description}

\section{자동 조사}
  \begin{itemize}
    \item \koTeX 과 마찬가지로 \cs{은} \cs{는} \cs{이} \cs{가}
      \cs{을} \cs{를} \cs{와} \cs{과} \cs{로} \cs{으로} \cs{라}
      \cs{이라}\,를 쓸 수 있다.
    \item \cs{ref} \cs{pageref} \cs{cite} 뒤에서만 정상 동작
    \item 아스키문자 뒤에서만 정상 동작
    \item 한글 뒤에는 \cs{jong} \cs{jung} \cs{rieul} 명령을
      첨가해 바로잡을 수 있다.\par
      \begin{itemize}
	\item[예:] \cs{cite\{hong\}}\cs{을} \ldots\\
		   \hskip1.3em\ \cs{bibitem[홍길동}\cs{jong]\{hong\}}
      \end{itemize}
  \end{itemize}

\section{한글 카운터}
\koTeX\ 패키지와 동일하다. 사용례: \cs{pagenumbering\{onum\}}
\begin{itemize}\leftskip=1cm \labelsep=1em \itemsep=0pt plus0pt
      \def\cs#1{\texttt{\bfseries #1}}
  \item[\cs{jaso}] ㄱ ㄴ ㄷ ㄹ ㅁ ㅂ ㅅ ㅇ ㅈ ㅊ ㅋ ㅌ ㅍ ㅎ
  \item[\cs{gana}] 가 나 다 라 마 바 사 아 자 차 카 타 파 하
  \item[\cs{ojaso}] ㉠ ㉡ ㉢ ㉣ ㉤ ㉥ ㉦ ㉧ ㉨ ㉩ ㉪ ㉫ ㉬ ㉭
  \item[\cs{ogana}] ㉮ ㉯ ㉰ ㉱ ㉲ ㉳ ㉴ ㉵ ㉶ ㉷ ㉸ ㉹ ㉺ ㉻
  \item[\cs{pjaso}] ㈀ ㈁ ㈂ ㈃ ㈄ ㈅ ㈆ ㈇ ㈈ ㈉ ㈊ ㈋ ㈌ ㈍
  \item[\cs{pgana}] ㈎ ㈏ ㈐ ㈑ ㈒ ㈓ ㈔ ㈕ ㈖ ㈗ ㈘ ㈙ ㈚ ㈛
  \item[\cs{onum}]  ① ② ③ ④ ⑤ ⑥ ⑦ ⑧ ⑨ ⑩ ⑪ ⑫ ⑬ ⑭ ⑮
  \item[\cs{pnum}] ⑴ ⑵ ⑶ ⑷ ⑸ ⑹ ⑺ ⑻ ⑼ ⑽ ⑾ ⑿ ⒀ ⒁ ⒂
  \item[\cs{oeng}] ⓐ ⓑ ⓒ ⓓ ⓔ ⓕ ⓖ ⓗ ⓘ ⓙ ⓚ ⓛ $\cdots$ ⓩ
  \item[\cs{peng}] ⒜ ⒝ ⒞ ⒟ ⒠ ⒡ ⒢ ⒣ ⒤ ⒥ ⒦ ⒧ $\cdots$ ⒵
  \item[\cs{hnum}] 하나 둘 셋 넷 다섯 여섯 일곱 여덟 아홉 열 열하나 $\cdots$ 스물넷
  \item[\cs{Hnum}] 첫 둘 셋 넷 다섯 여섯 일곱 여덟 아홉 열 열한 $\cdots$ 스물넷
  \item[\cs{hroman}] ⅰ ⅱ ⅲ ⅳ ⅴ ⅵ ⅶ ⅷ ⅸ ⅹ ⅺ ⅻ
  \item[\cs{hRoman}] Ⅰ Ⅱ Ⅲ Ⅳ Ⅴ Ⅵ Ⅶ Ⅷ Ⅸ Ⅹ Ⅺ Ⅻ
  \item[\cs{hNum}] 일 이 삼 사 오 육 칠 팔 구 십 십일 십이 $\cdots$ 이십사
  \item[\cs{hanjanum}] 一 二 三 四 五 六 七 八 九 十 十一 十二 $\cdots$ 二十四
\end{itemize}

\section{License}
\begin{itemize}
  \item GPL~--- \verb|cjkutf8-*| 파일의 라이선스는 CJK 패키지와
    같을 수밖에 없다.
  \item LPPL~--- \verb|ko*| 파일들은 \koTeX\ 패키지에서 유래한다.
  %\item \cjkko\ CVS~--- \url{http://cvs.ktug.or.kr/viewcvs/ko.TeX/cjk-ko/}
\end{itemize}

\begin{flushright}
  \fboxsep=-\fboxrule
  \fbox{\vbox to1em{\hbox to1em{\hss}\vss}}
\end{flushright}

\end{document}
